

	\documentclass[10pt,parskip=half,
	toc=sectionentrywithdots,
	bibliography=totocnumbered,
	captions=tableheading,numbers=noendperiod]{scrartcl}

    \usepackage[T1]{fontenc} % Nicer default font (+ math font) than Computer Modern for most use cases
    \usepackage{mathpazo}
    \usepackage{graphicx}
    \usepackage[skip=3pt]{caption}
    \usepackage{adjustbox} % Used to constrain images to a maximum size
    \usepackage[table]{xcolor} % Allow colors to be defined
    \usepackage{enumerate} % Needed for markdown enumerations to work
    \usepackage{amsmath} % Equations
    \usepackage{amssymb} % Equations
    \usepackage{textcomp} % defines textquotesingle
    % Hack from http://tex.stackexchange.com/a/47451/13684:
    \AtBeginDocument{%
        \def\PYZsq{\textquotesingle}% Upright quotes in Pygmentized code
    }
    \usepackage{upquote} % Upright quotes for verbatim code
    \usepackage{eurosym} % defines \euro
    \usepackage[mathletters]{ucs} % Extended unicode (utf-8) support
    \usepackage[utf8x]{inputenc} % Allow utf-8 characters in the tex document
    \usepackage{fancyvrb} % verbatim replacement that allows latex
    \usepackage{grffile} % extends the file name processing of package graphics
                         % to support a larger range
    % The hyperref package gives us a pdf with properly built
    % internal navigation ('pdf bookmarks' for the table of contents,
    % internal cross-reference links, web links for URLs, etc.)
    \usepackage{hyperref}
    \usepackage{longtable} % longtable support required by pandoc >1.10
    \usepackage{booktabs}  % table support for pandoc > 1.12.2
    \usepackage[inline]{enumitem} % IRkernel/repr support (it uses the enumerate* environment)
    \usepackage[normalem]{ulem} % ulem is needed to support strikethroughs (\sout)
                                % normalem makes italics be italics, not underlines

    \usepackage{translations}
	\usepackage{microtype} % improves the spacing between words and letters
	\usepackage{placeins} % placement of figures
    % could use \usepackage[section]{placeins} but placing in subsection in command section
	% Places the float at precisely the location in the LaTeX code (with H)
	\usepackage{float}
	\usepackage[colorinlistoftodos,obeyFinal,textwidth=.8in]{todonotes} % to mark to-dos
	% number figures, tables and equations by section
	\usepackage{chngcntr}
	% header/footer
	\usepackage[footsepline=0.25pt]{scrlayer-scrpage}

	% bibliography formatting
	\usepackage[numbers, square, super, sort&compress]{natbib}
	% hyperlink doi's
	\usepackage{doi}

    % define a code float
    \usepackage{newfloat} % to define a new float types
    \DeclareFloatingEnvironment[
        fileext=frm,placement={!ht},
        within=section,name=Code]{codecell}
    \DeclareFloatingEnvironment[
        fileext=frm,placement={!ht},
        within=section,name=Text]{textcell}
    \DeclareFloatingEnvironment[
        fileext=frm,placement={!ht},
        within=section,name=Text]{errorcell}

    \usepackage{listings} % a package for wrapping code in a box
    \usepackage[framemethod=tikz]{mdframed} % to fram code

% Pygments definitions

\makeatletter
\def\PY@reset{\let\PY@it=\relax \let\PY@bf=\relax%
    \let\PY@ul=\relax \let\PY@tc=\relax%
    \let\PY@bc=\relax \let\PY@ff=\relax}
\def\PY@tok#1{\csname PY@tok@#1\endcsname}
\def\PY@toks#1+{\ifx\relax#1\empty\else%
    \PY@tok{#1}\expandafter\PY@toks\fi}
\def\PY@do#1{\PY@bc{\PY@tc{\PY@ul{%
    \PY@it{\PY@bf{\PY@ff{#1}}}}}}}
\def\PY#1#2{\PY@reset\PY@toks#1+\relax+\PY@do{#2}}

\expandafter\def\csname PY@tok@w\endcsname{\def\PY@tc##1{\textcolor[rgb]{0.73,0.73,0.73}{##1}}}
\expandafter\def\csname PY@tok@c\endcsname{\let\PY@it=\textit\def\PY@tc##1{\textcolor[rgb]{0.25,0.50,0.50}{##1}}}
\expandafter\def\csname PY@tok@cp\endcsname{\def\PY@tc##1{\textcolor[rgb]{0.74,0.48,0.00}{##1}}}
\expandafter\def\csname PY@tok@k\endcsname{\let\PY@bf=\textbf\def\PY@tc##1{\textcolor[rgb]{0.00,0.50,0.00}{##1}}}
\expandafter\def\csname PY@tok@kp\endcsname{\def\PY@tc##1{\textcolor[rgb]{0.00,0.50,0.00}{##1}}}
\expandafter\def\csname PY@tok@kt\endcsname{\def\PY@tc##1{\textcolor[rgb]{0.69,0.00,0.25}{##1}}}
\expandafter\def\csname PY@tok@o\endcsname{\def\PY@tc##1{\textcolor[rgb]{0.40,0.40,0.40}{##1}}}
\expandafter\def\csname PY@tok@ow\endcsname{\let\PY@bf=\textbf\def\PY@tc##1{\textcolor[rgb]{0.67,0.13,1.00}{##1}}}
\expandafter\def\csname PY@tok@nb\endcsname{\def\PY@tc##1{\textcolor[rgb]{0.00,0.50,0.00}{##1}}}
\expandafter\def\csname PY@tok@nf\endcsname{\def\PY@tc##1{\textcolor[rgb]{0.00,0.00,1.00}{##1}}}
\expandafter\def\csname PY@tok@nc\endcsname{\let\PY@bf=\textbf\def\PY@tc##1{\textcolor[rgb]{0.00,0.00,1.00}{##1}}}
\expandafter\def\csname PY@tok@nn\endcsname{\let\PY@bf=\textbf\def\PY@tc##1{\textcolor[rgb]{0.00,0.00,1.00}{##1}}}
\expandafter\def\csname PY@tok@ne\endcsname{\let\PY@bf=\textbf\def\PY@tc##1{\textcolor[rgb]{0.82,0.25,0.23}{##1}}}
\expandafter\def\csname PY@tok@nv\endcsname{\def\PY@tc##1{\textcolor[rgb]{0.10,0.09,0.49}{##1}}}
\expandafter\def\csname PY@tok@no\endcsname{\def\PY@tc##1{\textcolor[rgb]{0.53,0.00,0.00}{##1}}}
\expandafter\def\csname PY@tok@nl\endcsname{\def\PY@tc##1{\textcolor[rgb]{0.63,0.63,0.00}{##1}}}
\expandafter\def\csname PY@tok@ni\endcsname{\let\PY@bf=\textbf\def\PY@tc##1{\textcolor[rgb]{0.60,0.60,0.60}{##1}}}
\expandafter\def\csname PY@tok@na\endcsname{\def\PY@tc##1{\textcolor[rgb]{0.49,0.56,0.16}{##1}}}
\expandafter\def\csname PY@tok@nt\endcsname{\let\PY@bf=\textbf\def\PY@tc##1{\textcolor[rgb]{0.00,0.50,0.00}{##1}}}
\expandafter\def\csname PY@tok@nd\endcsname{\def\PY@tc##1{\textcolor[rgb]{0.67,0.13,1.00}{##1}}}
\expandafter\def\csname PY@tok@s\endcsname{\def\PY@tc##1{\textcolor[rgb]{0.73,0.13,0.13}{##1}}}
\expandafter\def\csname PY@tok@sd\endcsname{\let\PY@it=\textit\def\PY@tc##1{\textcolor[rgb]{0.73,0.13,0.13}{##1}}}
\expandafter\def\csname PY@tok@si\endcsname{\let\PY@bf=\textbf\def\PY@tc##1{\textcolor[rgb]{0.73,0.40,0.53}{##1}}}
\expandafter\def\csname PY@tok@se\endcsname{\let\PY@bf=\textbf\def\PY@tc##1{\textcolor[rgb]{0.73,0.40,0.13}{##1}}}
\expandafter\def\csname PY@tok@sr\endcsname{\def\PY@tc##1{\textcolor[rgb]{0.73,0.40,0.53}{##1}}}
\expandafter\def\csname PY@tok@ss\endcsname{\def\PY@tc##1{\textcolor[rgb]{0.10,0.09,0.49}{##1}}}
\expandafter\def\csname PY@tok@sx\endcsname{\def\PY@tc##1{\textcolor[rgb]{0.00,0.50,0.00}{##1}}}
\expandafter\def\csname PY@tok@m\endcsname{\def\PY@tc##1{\textcolor[rgb]{0.40,0.40,0.40}{##1}}}
\expandafter\def\csname PY@tok@gh\endcsname{\let\PY@bf=\textbf\def\PY@tc##1{\textcolor[rgb]{0.00,0.00,0.50}{##1}}}
\expandafter\def\csname PY@tok@gu\endcsname{\let\PY@bf=\textbf\def\PY@tc##1{\textcolor[rgb]{0.50,0.00,0.50}{##1}}}
\expandafter\def\csname PY@tok@gd\endcsname{\def\PY@tc##1{\textcolor[rgb]{0.63,0.00,0.00}{##1}}}
\expandafter\def\csname PY@tok@gi\endcsname{\def\PY@tc##1{\textcolor[rgb]{0.00,0.63,0.00}{##1}}}
\expandafter\def\csname PY@tok@gr\endcsname{\def\PY@tc##1{\textcolor[rgb]{1.00,0.00,0.00}{##1}}}
\expandafter\def\csname PY@tok@ge\endcsname{\let\PY@it=\textit}
\expandafter\def\csname PY@tok@gs\endcsname{\let\PY@bf=\textbf}
\expandafter\def\csname PY@tok@gp\endcsname{\let\PY@bf=\textbf\def\PY@tc##1{\textcolor[rgb]{0.00,0.00,0.50}{##1}}}
\expandafter\def\csname PY@tok@go\endcsname{\def\PY@tc##1{\textcolor[rgb]{0.53,0.53,0.53}{##1}}}
\expandafter\def\csname PY@tok@gt\endcsname{\def\PY@tc##1{\textcolor[rgb]{0.00,0.27,0.87}{##1}}}
\expandafter\def\csname PY@tok@err\endcsname{\def\PY@bc##1{\setlength{\fboxsep}{0pt}\fcolorbox[rgb]{1.00,0.00,0.00}{1,1,1}{\strut ##1}}}
\expandafter\def\csname PY@tok@kc\endcsname{\let\PY@bf=\textbf\def\PY@tc##1{\textcolor[rgb]{0.00,0.50,0.00}{##1}}}
\expandafter\def\csname PY@tok@kd\endcsname{\let\PY@bf=\textbf\def\PY@tc##1{\textcolor[rgb]{0.00,0.50,0.00}{##1}}}
\expandafter\def\csname PY@tok@kn\endcsname{\let\PY@bf=\textbf\def\PY@tc##1{\textcolor[rgb]{0.00,0.50,0.00}{##1}}}
\expandafter\def\csname PY@tok@kr\endcsname{\let\PY@bf=\textbf\def\PY@tc##1{\textcolor[rgb]{0.00,0.50,0.00}{##1}}}
\expandafter\def\csname PY@tok@bp\endcsname{\def\PY@tc##1{\textcolor[rgb]{0.00,0.50,0.00}{##1}}}
\expandafter\def\csname PY@tok@fm\endcsname{\def\PY@tc##1{\textcolor[rgb]{0.00,0.00,1.00}{##1}}}
\expandafter\def\csname PY@tok@vc\endcsname{\def\PY@tc##1{\textcolor[rgb]{0.10,0.09,0.49}{##1}}}
\expandafter\def\csname PY@tok@vg\endcsname{\def\PY@tc##1{\textcolor[rgb]{0.10,0.09,0.49}{##1}}}
\expandafter\def\csname PY@tok@vi\endcsname{\def\PY@tc##1{\textcolor[rgb]{0.10,0.09,0.49}{##1}}}
\expandafter\def\csname PY@tok@vm\endcsname{\def\PY@tc##1{\textcolor[rgb]{0.10,0.09,0.49}{##1}}}
\expandafter\def\csname PY@tok@sa\endcsname{\def\PY@tc##1{\textcolor[rgb]{0.73,0.13,0.13}{##1}}}
\expandafter\def\csname PY@tok@sb\endcsname{\def\PY@tc##1{\textcolor[rgb]{0.73,0.13,0.13}{##1}}}
\expandafter\def\csname PY@tok@sc\endcsname{\def\PY@tc##1{\textcolor[rgb]{0.73,0.13,0.13}{##1}}}
\expandafter\def\csname PY@tok@dl\endcsname{\def\PY@tc##1{\textcolor[rgb]{0.73,0.13,0.13}{##1}}}
\expandafter\def\csname PY@tok@s2\endcsname{\def\PY@tc##1{\textcolor[rgb]{0.73,0.13,0.13}{##1}}}
\expandafter\def\csname PY@tok@sh\endcsname{\def\PY@tc##1{\textcolor[rgb]{0.73,0.13,0.13}{##1}}}
\expandafter\def\csname PY@tok@s1\endcsname{\def\PY@tc##1{\textcolor[rgb]{0.73,0.13,0.13}{##1}}}
\expandafter\def\csname PY@tok@mb\endcsname{\def\PY@tc##1{\textcolor[rgb]{0.40,0.40,0.40}{##1}}}
\expandafter\def\csname PY@tok@mf\endcsname{\def\PY@tc##1{\textcolor[rgb]{0.40,0.40,0.40}{##1}}}
\expandafter\def\csname PY@tok@mh\endcsname{\def\PY@tc##1{\textcolor[rgb]{0.40,0.40,0.40}{##1}}}
\expandafter\def\csname PY@tok@mi\endcsname{\def\PY@tc##1{\textcolor[rgb]{0.40,0.40,0.40}{##1}}}
\expandafter\def\csname PY@tok@il\endcsname{\def\PY@tc##1{\textcolor[rgb]{0.40,0.40,0.40}{##1}}}
\expandafter\def\csname PY@tok@mo\endcsname{\def\PY@tc##1{\textcolor[rgb]{0.40,0.40,0.40}{##1}}}
\expandafter\def\csname PY@tok@ch\endcsname{\let\PY@it=\textit\def\PY@tc##1{\textcolor[rgb]{0.25,0.50,0.50}{##1}}}
\expandafter\def\csname PY@tok@cm\endcsname{\let\PY@it=\textit\def\PY@tc##1{\textcolor[rgb]{0.25,0.50,0.50}{##1}}}
\expandafter\def\csname PY@tok@cpf\endcsname{\let\PY@it=\textit\def\PY@tc##1{\textcolor[rgb]{0.25,0.50,0.50}{##1}}}
\expandafter\def\csname PY@tok@c1\endcsname{\let\PY@it=\textit\def\PY@tc##1{\textcolor[rgb]{0.25,0.50,0.50}{##1}}}
\expandafter\def\csname PY@tok@cs\endcsname{\let\PY@it=\textit\def\PY@tc##1{\textcolor[rgb]{0.25,0.50,0.50}{##1}}}

\def\PYZbs{\char`\\}
\def\PYZus{\char`\_}
\def\PYZob{\char`\{}
\def\PYZcb{\char`\}}
\def\PYZca{\char`\^}
\def\PYZam{\char`\&}
\def\PYZlt{\char`\<}
\def\PYZgt{\char`\>}
\def\PYZsh{\char`\#}
\def\PYZpc{\char`\%}
\def\PYZdl{\char`\$}
\def\PYZhy{\char`\-}
\def\PYZsq{\char`\'}
\def\PYZdq{\char`\"}
\def\PYZti{\char`\~}
% for compatibility with earlier versions
\def\PYZat{@}
\def\PYZlb{[}
\def\PYZrb{]}
\makeatother

% ANSI colors
\definecolor{ansi-black}{HTML}{3E424D}
\definecolor{ansi-black-intense}{HTML}{282C36}
\definecolor{ansi-red}{HTML}{E75C58}
\definecolor{ansi-red-intense}{HTML}{B22B31}
\definecolor{ansi-green}{HTML}{00A250}
\definecolor{ansi-green-intense}{HTML}{007427}
\definecolor{ansi-yellow}{HTML}{DDB62B}
\definecolor{ansi-yellow-intense}{HTML}{B27D12}
\definecolor{ansi-blue}{HTML}{208FFB}
\definecolor{ansi-blue-intense}{HTML}{0065CA}
\definecolor{ansi-magenta}{HTML}{D160C4}
\definecolor{ansi-magenta-intense}{HTML}{A03196}
\definecolor{ansi-cyan}{HTML}{60C6C8}
\definecolor{ansi-cyan-intense}{HTML}{258F8F}
\definecolor{ansi-white}{HTML}{C5C1B4}
\definecolor{ansi-white-intense}{HTML}{A1A6B2}

% commands and environments needed by pandoc snippets
% extracted from the output of `pandoc -s`
\providecommand{\tightlist}{%
  \setlength{\itemsep}{0pt}\setlength{\parskip}{0pt}}
\DefineVerbatimEnvironment{Highlighting}{Verbatim}{commandchars=\\\{\}}
% Add ',fontsize=\small' for more characters per line
\newenvironment{Shaded}{}{}
\newcommand{\KeywordTok}[1]{\textcolor[rgb]{0.00,0.44,0.13}{\textbf{{#1}}}}
\newcommand{\DataTypeTok}[1]{\textcolor[rgb]{0.56,0.13,0.00}{{#1}}}
\newcommand{\DecValTok}[1]{\textcolor[rgb]{0.25,0.63,0.44}{{#1}}}
\newcommand{\BaseNTok}[1]{\textcolor[rgb]{0.25,0.63,0.44}{{#1}}}
\newcommand{\FloatTok}[1]{\textcolor[rgb]{0.25,0.63,0.44}{{#1}}}
\newcommand{\CharTok}[1]{\textcolor[rgb]{0.25,0.44,0.63}{{#1}}}
\newcommand{\StringTok}[1]{\textcolor[rgb]{0.25,0.44,0.63}{{#1}}}
\newcommand{\CommentTok}[1]{\textcolor[rgb]{0.38,0.63,0.69}{\textit{{#1}}}}
\newcommand{\OtherTok}[1]{\textcolor[rgb]{0.00,0.44,0.13}{{#1}}}
\newcommand{\AlertTok}[1]{\textcolor[rgb]{1.00,0.00,0.00}{\textbf{{#1}}}}
\newcommand{\FunctionTok}[1]{\textcolor[rgb]{0.02,0.16,0.49}{{#1}}}
\newcommand{\RegionMarkerTok}[1]{{#1}}
\newcommand{\ErrorTok}[1]{\textcolor[rgb]{1.00,0.00,0.00}{\textbf{{#1}}}}
\newcommand{\NormalTok}[1]{{#1}}

% Additional commands for more recent versions of Pandoc
\newcommand{\ConstantTok}[1]{\textcolor[rgb]{0.53,0.00,0.00}{{#1}}}
\newcommand{\SpecialCharTok}[1]{\textcolor[rgb]{0.25,0.44,0.63}{{#1}}}
\newcommand{\VerbatimStringTok}[1]{\textcolor[rgb]{0.25,0.44,0.63}{{#1}}}
\newcommand{\SpecialStringTok}[1]{\textcolor[rgb]{0.73,0.40,0.53}{{#1}}}
\newcommand{\ImportTok}[1]{{#1}}
\newcommand{\DocumentationTok}[1]{\textcolor[rgb]{0.73,0.13,0.13}{\textit{{#1}}}}
\newcommand{\AnnotationTok}[1]{\textcolor[rgb]{0.38,0.63,0.69}{\textbf{\textit{{#1}}}}}
\newcommand{\CommentVarTok}[1]{\textcolor[rgb]{0.38,0.63,0.69}{\textbf{\textit{{#1}}}}}
\newcommand{\VariableTok}[1]{\textcolor[rgb]{0.10,0.09,0.49}{{#1}}}
\newcommand{\ControlFlowTok}[1]{\textcolor[rgb]{0.00,0.44,0.13}{\textbf{{#1}}}}
\newcommand{\OperatorTok}[1]{\textcolor[rgb]{0.40,0.40,0.40}{{#1}}}
\newcommand{\BuiltInTok}[1]{{#1}}
\newcommand{\ExtensionTok}[1]{{#1}}
\newcommand{\PreprocessorTok}[1]{\textcolor[rgb]{0.74,0.48,0.00}{{#1}}}
\newcommand{\AttributeTok}[1]{\textcolor[rgb]{0.49,0.56,0.16}{{#1}}}
\newcommand{\InformationTok}[1]{\textcolor[rgb]{0.38,0.63,0.69}{\textbf{\textit{{#1}}}}}
\newcommand{\WarningTok}[1]{\textcolor[rgb]{0.38,0.63,0.69}{\textbf{\textit{{#1}}}}}

% Define a nice break command that doesn't care if a line doesn't already
% exist.
\def\br{\hspace*{\fill} \\* }

% Math Jax compatability definitions
\def\gt{>}
\def\lt{<}

    \setcounter{secnumdepth}{5}

    % Colors for the hyperref package
    \definecolor{urlcolor}{rgb}{0,.145,.698}
    \definecolor{linkcolor}{rgb}{.71,0.21,0.01}
    \definecolor{citecolor}{rgb}{.12,.54,.11}

\DeclareTranslationFallback{Author}{Author}
\DeclareTranslation{Portuges}{Author}{Autor}

\DeclareTranslationFallback{List of Codes}{List of Codes}
\DeclareTranslation{Catalan}{List of Codes}{Llista de Codis}
\DeclareTranslation{Danish}{List of Codes}{Liste over Koder}
\DeclareTranslation{German}{List of Codes}{Liste der Codes}
\DeclareTranslation{Spanish}{List of Codes}{Lista de C\'{o}digos}
\DeclareTranslation{French}{List of Codes}{Liste des Codes}
\DeclareTranslation{Italian}{List of Codes}{Elenco dei Codici}
\DeclareTranslation{Dutch}{List of Codes}{Lijst van Codes}
\DeclareTranslation{Portuges}{List of Codes}{Lista de C\'{o}digos}

\DeclareTranslationFallback{Supervisors}{Supervisors}
\DeclareTranslation{Catalan}{Supervisors}{Supervisors}
\DeclareTranslation{Danish}{Supervisors}{Vejledere}
\DeclareTranslation{German}{Supervisors}{Vorgesetzten}
\DeclareTranslation{Spanish}{Supervisors}{Supervisores}
\DeclareTranslation{French}{Supervisors}{Superviseurs}
\DeclareTranslation{Italian}{Supervisors}{Le autorit\`{a} di vigilanza}
\DeclareTranslation{Dutch}{Supervisors}{supervisors}
\DeclareTranslation{Portuguese}{Supervisors}{Supervisores}

\definecolor{codegreen}{rgb}{0,0.6,0}
\definecolor{codegray}{rgb}{0.5,0.5,0.5}
\definecolor{codepurple}{rgb}{0.58,0,0.82}
\definecolor{backcolour}{rgb}{0.95,0.95,0.95}

\lstdefinestyle{mystyle}{
    commentstyle=\color{codegreen},
    keywordstyle=\color{magenta},
    numberstyle=\tiny\color{codegray},
    stringstyle=\color{codepurple},
    basicstyle=\ttfamily,
    breakatwhitespace=false,
    keepspaces=true,
    numbers=left,
    numbersep=10pt,
    showspaces=false,
    showstringspaces=false,
    showtabs=false,
    tabsize=2,
    breaklines=true,
    literate={\-}{}{0\discretionary{-}{}{-}},
  postbreak=\mbox{\textcolor{red}{$\hookrightarrow$}\space},
}

\lstset{style=mystyle}

\surroundwithmdframed[
  hidealllines=true,
  backgroundcolor=backcolour,
  innerleftmargin=0pt,
  innerrightmargin=0pt,
  innertopmargin=0pt,
  innerbottommargin=0pt]{lstlisting}

 % Used to adjust the document margins
\usepackage{geometry}
\geometry{tmargin=1in,bmargin=1in,lmargin=1in,rmargin=1in,
nohead,includefoot,footskip=25pt}
% you can use showframe option to check the margins visually

	% ensure new section starts on new page
	\addtokomafont{section}{\clearpage}

    % Prevent overflowing lines due to hard-to-break entities
    \sloppy

    % Setup hyperref package
    \hypersetup{
      breaklinks=true,  % so long urls are correctly broken across lines
      colorlinks=true,
      urlcolor=urlcolor,
      linkcolor=linkcolor,
      citecolor=citecolor,
      }

    % ensure figures are placed within subsections
    \makeatletter
    \AtBeginDocument{%
      \expandafter\renewcommand\expandafter\subsection\expandafter
        {\expandafter\@fb@secFB\subsection}%
      \newcommand\@fb@secFB{\FloatBarrier
        \gdef\@fb@afterHHook{\@fb@topbarrier \gdef\@fb@afterHHook{}}}%
      \g@addto@macro\@afterheading{\@fb@afterHHook}%
      \gdef\@fb@afterHHook{}%
    }
    \makeatother

	% number figures, tables and equations by section
	\usepackage{chngcntr}
	\counterwithout{figure}{section}
	\counterwithout{table}{section}
	\counterwithout{equation}{section}
	\makeatletter
	\@addtoreset{table}{section}
	\@addtoreset{figure}{section}
	\@addtoreset{equation}{section}
	\makeatother
	\renewcommand\thetable{\thesection.\arabic{table}}
	\renewcommand\thefigure{\thesection.\arabic{figure}}
	\renewcommand\theequation{\thesection.\arabic{equation}}

        % set global options for float placement
        \makeatletter
          \providecommand*\setfloatlocations[2]{\@namedef{fps@#1}{#2}}
        \makeatother

    % align captions to left (indented)
	\captionsetup{justification=raggedright,
	singlelinecheck=false,format=hang,labelfont={it,bf}}

	% shift footer down so space between separation line
	\ModifyLayer[addvoffset=.6ex]{scrheadings.foot.odd}
	\ModifyLayer[addvoffset=.6ex]{scrheadings.foot.even}
	\ModifyLayer[addvoffset=.6ex]{scrheadings.foot.oneside}
	\ModifyLayer[addvoffset=.6ex]{plain.scrheadings.foot.odd}
	\ModifyLayer[addvoffset=.6ex]{plain.scrheadings.foot.even}
	\ModifyLayer[addvoffset=.6ex]{plain.scrheadings.foot.oneside}
	\pagestyle{scrheadings}
	\clearscrheadfoot{}
	\ifoot{\leftmark}
	\renewcommand{\sectionmark}[1]{\markleft{\thesection\ #1}}
	\ofoot{\pagemark}
	\cfoot{}

% clereref must be loaded after anything that changes the referencing system
\usepackage{cleveref}
\creflabelformat{equation}{#2#1#3}

% make the code float work with cleverref
\crefname{codecell}{code}{codes}
\Crefname{codecell}{code}{codes}
% make the text float work with cleverref
\crefname{textcell}{text}{texts}
\Crefname{textcell}{text}{texts}
% make the text float work with cleverref
\crefname{errorcell}{error}{errors}
\Crefname{errorcell}{error}{errors}

	\begin{document}

		\title{Notebook}
	\date{\today}
	\maketitle

		\begingroup
    \let\cleardoublepage\relax
    \let\clearpage\relax\tableofcontents\listoffigures\listoftables\listof{codecell}{\GetTranslation{List of Codes}}
    \endgroup

\subsubsection{NLP Refresher}\label{nlp-refresher}

\begin{lstlisting}[language=Python,numbers=left,xleftmargin=20pt,xrightmargin=5pt,belowskip=5pt,aboveskip=5pt]
text = ["System of the World. By Isaac Newton", "   Snow Crash  .  By Neal Stephenson ",
       " AFROFUTURISM. by     Ytasha L. Womack "]
\end{lstlisting}

\begin{lstlisting}[language=Python,numbers=left,xleftmargin=20pt,xrightmargin=5pt,belowskip=5pt,aboveskip=5pt]
strip_whitespace = [string.strip() for string in text]
\end{lstlisting}

\begin{lstlisting}[language=Python,numbers=left,xleftmargin=20pt,xrightmargin=5pt,belowskip=5pt,aboveskip=5pt]
strip_whitespace
\end{lstlisting}

\begin{lstlisting}[language={},postbreak={},numbers=none,xrightmargin=7pt,breakindent=0pt,aboveskip=5pt,belowskip=5pt]
['System of the World. By Isaac Newton',
 'Snow Crash  .  By Neal Stephenson',
 'AFROFUTURISM. by     Ytasha L. Womack']
\end{lstlisting}

\begin{lstlisting}[language=Python,numbers=left,xleftmargin=20pt,xrightmargin=5pt,belowskip=5pt,aboveskip=5pt]
strip_whitespace2 = [string.strip() for string in strip_whitespace]
strip_whitespace2
\end{lstlisting}

\begin{lstlisting}[language={},postbreak={},numbers=none,xrightmargin=7pt,breakindent=0pt,aboveskip=5pt,belowskip=5pt]
['System of the World. By Isaac Newton',
 'Snow Crash  .  By Neal Stephenson',
 'AFROFUTURISM. by     Ytasha L. Womack']
\end{lstlisting}

\begin{lstlisting}[language=Python,numbers=left,xleftmargin=20pt,xrightmargin=5pt,belowskip=5pt,aboveskip=5pt]
remove_periods = [string.replace(".","") for string in strip_whitespace]
\end{lstlisting}

\begin{lstlisting}[language=Python,numbers=left,xleftmargin=20pt,xrightmargin=5pt,belowskip=5pt,aboveskip=5pt]
remove_periods
\end{lstlisting}

\begin{lstlisting}[language={},postbreak={},numbers=none,xrightmargin=7pt,breakindent=0pt,aboveskip=5pt,belowskip=5pt]
['System of the World By Isaac Newton',
 'Snow Crash    By Neal Stephenson',
 'AFROFUTURISM by     Ytasha L Womack']
\end{lstlisting}

\begin{lstlisting}[language=Python,numbers=left,xleftmargin=20pt,xrightmargin=5pt,belowskip=5pt,aboveskip=5pt]
upper = [string.upper() for string in strip_whitespace]
\end{lstlisting}

\begin{lstlisting}[language=Python,numbers=left,xleftmargin=20pt,xrightmargin=5pt,belowskip=5pt,aboveskip=5pt]
upper
\end{lstlisting}

\begin{lstlisting}[language={},postbreak={},numbers=none,xrightmargin=7pt,breakindent=0pt,aboveskip=5pt,belowskip=5pt]
['SYSTEM OF THE WORLD. BY ISAAC NEWTON',
 'SNOW CRASH  .  BY NEAL STEPHENSON',
 'AFROFUTURISM. BY     YTASHA L. WOMACK']
\end{lstlisting}

\begin{lstlisting}[language=Python,numbers=left,xleftmargin=20pt,xrightmargin=5pt,belowskip=5pt,aboveskip=5pt]
import re
\end{lstlisting}

\begin{lstlisting}[language=Python,numbers=left,xleftmargin=20pt,xrightmargin=5pt,belowskip=5pt,aboveskip=5pt]
xs = [re.sub(r"[a-zA-Z]", "X", string) for string in strip_whitespace]
\end{lstlisting}

\begin{lstlisting}[language=Python,numbers=left,xleftmargin=20pt,xrightmargin=5pt,belowskip=5pt,aboveskip=5pt]
xs
\end{lstlisting}

\begin{lstlisting}[language={},postbreak={},numbers=none,xrightmargin=7pt,breakindent=0pt,aboveskip=5pt,belowskip=5pt]
['XXXXXX XX XXX XXXXX. XX XXXXX XXXXXX',
 'XXXX XXXXX  .  XX XXXX XXXXXXXXXX',
 'XXXXXXXXXXXX. XX     XXXXXX X. XXXXXX']
\end{lstlisting}

\textbf{REGEX TUTORIAL}

https://www.analyticsvidhya.com/blog/2015/06/regular-expression-python/

\subsubsection{Scraping}\label{scraping}

\begin{lstlisting}[language=Python,numbers=left,xleftmargin=20pt,xrightmargin=5pt,belowskip=5pt,aboveskip=5pt]
import requests
from bs4 import BeautifulSoup
\end{lstlisting}

\begin{lstlisting}[language=Python,numbers=left,xleftmargin=20pt,xrightmargin=5pt,belowskip=5pt,aboveskip=5pt]
url = 'https://www.analyticsvidhya.com/blog/2015/06/regular-expression-python/'
\end{lstlisting}

\begin{lstlisting}[language=Python,numbers=left,xleftmargin=20pt,xrightmargin=5pt,belowskip=5pt,aboveskip=5pt]
req = requests.get(url)
\end{lstlisting}

\begin{lstlisting}[language=Python,numbers=left,xleftmargin=20pt,xrightmargin=5pt,belowskip=5pt,aboveskip=5pt]
req
\end{lstlisting}

\begin{lstlisting}[language={},postbreak={},numbers=none,xrightmargin=7pt,breakindent=0pt,aboveskip=5pt,belowskip=5pt]
<Response [200]>
\end{lstlisting}

\begin{lstlisting}[language=Python,numbers=left,xleftmargin=20pt,xrightmargin=5pt,belowskip=5pt,aboveskip=5pt]
soup = BeautifulSoup(req.text, 'html.parser')
\end{lstlisting}

\begin{lstlisting}[language=Python,numbers=left,xleftmargin=20pt,xrightmargin=5pt,belowskip=5pt,aboveskip=5pt]
soup.text[40:50]
\end{lstlisting}

\begin{lstlisting}[language={},postbreak={},numbers=none,xrightmargin=7pt,breakindent=0pt,aboveskip=5pt,belowskip=5pt]
'lar Expres'
\end{lstlisting}

\begin{lstlisting}[language=Python,numbers=left,xleftmargin=20pt,xrightmargin=5pt,belowskip=5pt,aboveskip=5pt]
soup.find('h2')
\end{lstlisting}

\begin{lstlisting}[language={},postbreak={},numbers=none,xrightmargin=7pt,breakindent=0pt,aboveskip=5pt,belowskip=5pt]
<h2 class="site-outline">Learn everything about Analytics</h2>
\end{lstlisting}

\begin{lstlisting}[language=Python,numbers=left,xleftmargin=20pt,xrightmargin=5pt,belowskip=5pt,aboveskip=5pt]
heads = soup.find_all('h2')
\end{lstlisting}

\begin{lstlisting}[language=Python,numbers=left,xleftmargin=20pt,xrightmargin=5pt,belowskip=5pt,aboveskip=5pt]
len(heads)
\end{lstlisting}

\begin{lstlisting}[language={},postbreak={},numbers=none,xrightmargin=7pt,breakindent=0pt,aboveskip=5pt,belowskip=5pt]
6
\end{lstlisting}

\subsubsection{Basic NLP}\label{basic-nlp}

\begin{lstlisting}[language=Python,numbers=left,xleftmargin=20pt,xrightmargin=5pt,belowskip=5pt,aboveskip=5pt]
from nltk.tokenize import word_tokenize
import nltk
\end{lstlisting}

\begin{lstlisting}[language=Python,numbers=left,xleftmargin=20pt,xrightmargin=5pt,belowskip=5pt,aboveskip=5pt]
nltk.download('punkt')
\end{lstlisting}

\begin{lstlisting}[language={},postbreak={},numbers=none,xrightmargin=7pt,belowskip=5pt,aboveskip=5pt,breakindent=0pt]
[nltk_data] Downloading package punkt to /Users/NYCMath/nltk_data...
[nltk_data]   Package punkt is already up-to-date!

\end{lstlisting}

\begin{lstlisting}[language={},postbreak={},numbers=none,xrightmargin=7pt,breakindent=0pt,aboveskip=5pt,belowskip=5pt]
True
\end{lstlisting}

\begin{lstlisting}[language=Python,numbers=left,xleftmargin=20pt,xrightmargin=5pt,belowskip=5pt,aboveskip=5pt]
pgraph = soup.find('p').text
\end{lstlisting}

\begin{lstlisting}[language=Python,numbers=left,xleftmargin=20pt,xrightmargin=5pt,belowskip=5pt,aboveskip=5pt]
tokes = word_tokenize(pgraph)
\end{lstlisting}

\begin{lstlisting}[language=Python,numbers=left,xleftmargin=20pt,xrightmargin=5pt,belowskip=5pt,aboveskip=5pt]
tokes
\end{lstlisting}

\begin{lstlisting}[language={},postbreak={},numbers=none,xrightmargin=7pt,breakindent=0pt,aboveskip=5pt,belowskip=5pt]
['In',
 'last',
 'few',
 'years',
 ',',
 'there',
 'has',
 'been',
 'a',
 'dramatic',
 'shift',
 'in',
 'usage',
 'of',
 'general',
 'purpose',
 'programming',
 'languages',
 'for',
 'data',
 'science',
 'and',
 'machine',
 'learning',
 '.',
 'This',
 'was',
 'not',
 'always',
 'the',
 'case',
 '–',
 'a',
 'decade',
 'back',
 'this',
 'thought',
 'would',
 'have',
 'met',
 'a',
 'lot',
 'of',
 'skeptic',
 'eyes',
 '!']
\end{lstlisting}

\begin{lstlisting}[language=Python,numbers=left,xleftmargin=20pt,xrightmargin=5pt,belowskip=5pt,aboveskip=5pt]
sy = ['.', ',', '!', '-', '?','*', '–']
\end{lstlisting}

\begin{lstlisting}[language=Python,numbers=left,xleftmargin=20pt,xrightmargin=5pt,belowskip=5pt,aboveskip=5pt]
for word in tokes:
    if word not in sy:
        print(word)
    else:
        _
\end{lstlisting}

\begin{lstlisting}[language={},postbreak={},numbers=none,xrightmargin=7pt,belowskip=5pt,aboveskip=5pt,breakindent=0pt]
In
last
few
years
there
has
been
a
dramatic
shift
in
usage
of
general
purpose
programming
languages
for
data
science
and
machine
learning
This
was
not
always
the
case
a
decade
back
this
thought
would
have
met
a
lot
of
skeptic
eyes

\end{lstlisting}

\begin{lstlisting}[language=Python,numbers=left,xleftmargin=20pt,xrightmargin=5pt,belowskip=5pt,aboveskip=5pt]
from nltk.tokenize import sent_tokenize
\end{lstlisting}

\begin{lstlisting}[language=Python,numbers=left,xleftmargin=20pt,xrightmargin=5pt,belowskip=5pt,aboveskip=5pt]
sent_tokenize(pgraph)[0]
\end{lstlisting}

\begin{lstlisting}[language={},postbreak={},numbers=none,xrightmargin=7pt,breakindent=0pt,aboveskip=5pt,belowskip=5pt]
'In last few years, there has been a dramatic shift in usage of general purpose programming languages for data science and machine learning.'
\end{lstlisting}

\begin{lstlisting}[language=Python,numbers=left,xleftmargin=20pt,xrightmargin=5pt,belowskip=5pt,aboveskip=5pt]
from nltk.corpus import stopwords
\end{lstlisting}

\begin{lstlisting}[language=Python,numbers=left,xleftmargin=20pt,xrightmargin=5pt,belowskip=5pt,aboveskip=5pt]
stop_words = stopwords.words('english')
\end{lstlisting}

\begin{lstlisting}[language=Python,numbers=left,xleftmargin=20pt,xrightmargin=5pt,belowskip=5pt,aboveskip=5pt]
stop_words[:6]
\end{lstlisting}

\begin{lstlisting}[language={},postbreak={},numbers=none,xrightmargin=7pt,breakindent=0pt,aboveskip=5pt,belowskip=5pt]
['i', 'me', 'my', 'myself', 'we', 'our']
\end{lstlisting}

\begin{lstlisting}[language=Python,numbers=left,xleftmargin=20pt,xrightmargin=5pt,belowskip=5pt,aboveskip=5pt]
[word for word in tokes if word not in stop_words]
\end{lstlisting}

\begin{lstlisting}[language={},postbreak={},numbers=none,xrightmargin=7pt,breakindent=0pt,aboveskip=5pt,belowskip=5pt]
['In',
 'last',
 'years',
 ',',
 'dramatic',
 'shift',
 'usage',
 'general',
 'purpose',
 'programming',
 'languages',
 'data',
 'science',
 'machine',
 'learning',
 '.',
 'This',
 'always',
 'case',
 '–',
 'decade',
 'back',
 'thought',
 'would',
 'met',
 'lot',
 'skeptic',
 'eyes',
 '!']
\end{lstlisting}

\begin{lstlisting}[language=Python,numbers=left,xleftmargin=20pt,xrightmargin=5pt,belowskip=5pt,aboveskip=5pt]
stop_words[:10]
\end{lstlisting}

\begin{lstlisting}[language={},postbreak={},numbers=none,xrightmargin=7pt,breakindent=0pt,aboveskip=5pt,belowskip=5pt]
['i', 'me', 'my', 'myself', 'we', 'our', 'ours', 'ourselves', 'you', "you're"]
\end{lstlisting}

\begin{lstlisting}[language=Python,numbers=left,xleftmargin=20pt,xrightmargin=5pt,belowskip=5pt,aboveskip=5pt]
#stemming
from nltk.stem.porter import PorterStemmer
\end{lstlisting}

\begin{lstlisting}[language=Python,numbers=left,xleftmargin=20pt,xrightmargin=5pt,belowskip=5pt,aboveskip=5pt]
porter = PorterStemmer()
\end{lstlisting}

\begin{lstlisting}[language=Python,numbers=left,xleftmargin=20pt,xrightmargin=5pt,belowskip=5pt,aboveskip=5pt]
[porter.stem(word) for word in tokes]
\end{lstlisting}

\begin{lstlisting}[language={},postbreak={},numbers=none,xrightmargin=7pt,breakindent=0pt,aboveskip=5pt,belowskip=5pt]
['In',
 'last',
 'few',
 'year',
 ',',
 'there',
 'ha',
 'been',
 'a',
 'dramat',
 'shift',
 'in',
 'usag',
 'of',
 'gener',
 'purpos',
 'program',
 'languag',
 'for',
 'data',
 'scienc',
 'and',
 'machin',
 'learn',
 '.',
 'thi',
 'wa',
 'not',
 'alway',
 'the',
 'case',
 '–',
 'a',
 'decad',
 'back',
 'thi',
 'thought',
 'would',
 'have',
 'met',
 'a',
 'lot',
 'of',
 'skeptic',
 'eye',
 '!']
\end{lstlisting}

\begin{lstlisting}[language=Python,numbers=left,xleftmargin=20pt,xrightmargin=5pt,belowskip=5pt,aboveskip=5pt]
from nltk import pos_tag
\end{lstlisting}

\begin{lstlisting}[language=Python,numbers=left,xleftmargin=20pt,xrightmargin=5pt,belowskip=5pt,aboveskip=5pt]
text_tagged = pos_tag(tokes)
\end{lstlisting}

\begin{lstlisting}[language=Python,numbers=left,xleftmargin=20pt,xrightmargin=5pt,belowskip=5pt,aboveskip=5pt]
text_tagged
\end{lstlisting}

\begin{lstlisting}[language={},postbreak={},numbers=none,xrightmargin=7pt,breakindent=0pt,aboveskip=5pt,belowskip=5pt]
[('In', 'IN'),
 ('last', 'JJ'),
 ('few', 'JJ'),
 ('years', 'NNS'),
 (',', ','),
 ('there', 'EX'),
 ('has', 'VBZ'),
 ('been', 'VBN'),
 ('a', 'DT'),
 ('dramatic', 'JJ'),
 ('shift', 'NN'),
 ('in', 'IN'),
 ('usage', 'NN'),
 ('of', 'IN'),
 ('general', 'JJ'),
 ('purpose', 'NN'),
 ('programming', 'NN'),
 ('languages', 'NNS'),
 ('for', 'IN'),
 ('data', 'NNS'),
 ('science', 'NN'),
 ('and', 'CC'),
 ('machine', 'NN'),
 ('learning', 'NN'),
 ('.', '.'),
 ('This', 'DT'),
 ('was', 'VBD'),
 ('not', 'RB'),
 ('always', 'RB'),
 ('the', 'DT'),
 ('case', 'NN'),
 ('–', 'VBZ'),
 ('a', 'DT'),
 ('decade', 'NN'),
 ('back', 'RB'),
 ('this', 'DT'),
 ('thought', 'NN'),
 ('would', 'MD'),
 ('have', 'VB'),
 ('met', 'VBN'),
 ('a', 'DT'),
 ('lot', 'NN'),
 ('of', 'IN'),
 ('skeptic', 'JJ'),
 ('eyes', 'NNS'),
 ('!', '.')]
\end{lstlisting}

\begin{lstlisting}[language=Python,numbers=left,xleftmargin=20pt,xrightmargin=5pt,belowskip=5pt,aboveskip=5pt]
[word for word, tag in text_tagged if tag in ['NN', 'NNS']]
\end{lstlisting}

\begin{lstlisting}[language={},postbreak={},numbers=none,xrightmargin=7pt,breakindent=0pt,aboveskip=5pt,belowskip=5pt]
['years',
 'shift',
 'usage',
 'purpose',
 'programming',
 'languages',
 'data',
 'science',
 'machine',
 'learning',
 'case',
 'decade',
 'thought',
 'lot',
 'eyes']
\end{lstlisting}

\begin{lstlisting}[language=Python,numbers=left,xleftmargin=20pt,xrightmargin=5pt,belowskip=5pt,aboveskip=5pt]
tweets = ["we are more worried about what we can lose than what we feel",
         "it's really cool to say I hate you. But it's not cool to say I love you. Love has a stigma",
         "Instead of doing what you feel you just do what other people think you should do"]
\end{lstlisting}

\begin{lstlisting}[language=Python,numbers=left,xleftmargin=20pt,xrightmargin=5pt,belowskip=5pt,aboveskip=5pt]
tagged_tweets = []
for tweet in tweets:
    tweet_tag = pos_tag(word_tokenize(tweet))
    tagged_tweets.append([tag for word, tag in tweet_tag])
\end{lstlisting}

\begin{lstlisting}[language=Python,numbers=left,xleftmargin=20pt,xrightmargin=5pt,belowskip=5pt,aboveskip=5pt]
tagged_tweets[2][:5]
\end{lstlisting}

\begin{lstlisting}[language={},postbreak={},numbers=none,xrightmargin=7pt,breakindent=0pt,aboveskip=5pt,belowskip=5pt]
['RB', 'IN', 'VBG', 'WP', 'PRP']
\end{lstlisting}

\begin{lstlisting}[language=Python,numbers=left,xleftmargin=20pt,xrightmargin=5pt,belowskip=5pt,aboveskip=5pt]
from sklearn.preprocessing import MultiLabelBinarizer
\end{lstlisting}

\begin{lstlisting}[language=Python,numbers=left,xleftmargin=20pt,xrightmargin=5pt,belowskip=5pt,aboveskip=5pt]
one_hot_multi = MultiLabelBinarizer()
\end{lstlisting}

\begin{lstlisting}[language=Python,numbers=left,xleftmargin=20pt,xrightmargin=5pt,belowskip=5pt,aboveskip=5pt]
one_hot_multi.fit_transform(tagged_tweets)
\end{lstlisting}

\begin{lstlisting}[language={},postbreak={},numbers=none,xrightmargin=7pt,breakindent=0pt,aboveskip=5pt,belowskip=5pt]
array([[0, 0, 0, 1, 1, 1, 0, 0, 1, 0, 1, 0, 1, 0, 1, 0, 1],
       [1, 1, 1, 0, 1, 0, 1, 0, 1, 1, 0, 1, 1, 0, 1, 1, 0],
       [0, 0, 0, 1, 1, 1, 0, 1, 1, 1, 0, 0, 1, 1, 1, 0, 1]])
\end{lstlisting}

\begin{lstlisting}[language=Python,numbers=left,xleftmargin=20pt,xrightmargin=5pt,belowskip=5pt,aboveskip=5pt]
one_hot_multi.classes_
\end{lstlisting}

\begin{lstlisting}[language={},postbreak={},numbers=none,xrightmargin=7pt,breakindent=0pt,aboveskip=5pt,belowskip=5pt]
array(['.', 'CC', 'DT', 'IN', 'JJ', 'MD', 'NN', 'NNS', 'PRP', 'RB', 'RBR',
       'TO', 'VB', 'VBG', 'VBP', 'VBZ', 'WP'], dtype=object)
\end{lstlisting}

\subsubsection{CountVectorizer}\label{countvectorizer}

\begin{lstlisting}[language=Python,numbers=left,xleftmargin=20pt,xrightmargin=5pt,belowskip=5pt,aboveskip=5pt]
import numpy as np
\end{lstlisting}

\begin{lstlisting}[language=Python,numbers=left,xleftmargin=20pt,xrightmargin=5pt,belowskip=5pt,aboveskip=5pt]
from sklearn.feature_extraction.text import CountVectorizer
\end{lstlisting}

\begin{lstlisting}[language=Python,numbers=left,xleftmargin=20pt,xrightmargin=5pt,belowskip=5pt,aboveskip=5pt]
text_data = np.array(['I like Cardi B. ', 'Tribeca is a strange place.', ' Germany is where they make volkswagen cars.'])
\end{lstlisting}

\begin{lstlisting}[language=Python,numbers=left,xleftmargin=20pt,xrightmargin=5pt,belowskip=5pt,aboveskip=5pt]
count = CountVectorizer()
\end{lstlisting}

\begin{lstlisting}[language=Python,numbers=left,xleftmargin=20pt,xrightmargin=5pt,belowskip=5pt,aboveskip=5pt]
bag_of_words = count.fit_transform(text_data)
\end{lstlisting}

\begin{lstlisting}[language=Python,numbers=left,xleftmargin=20pt,xrightmargin=5pt,belowskip=5pt,aboveskip=5pt]
count.get_feature_names()
\end{lstlisting}

\begin{lstlisting}[language={},postbreak={},numbers=none,xrightmargin=7pt,breakindent=0pt,aboveskip=5pt,belowskip=5pt]
['cardi',
 'cars',
 'germany',
 'is',
 'like',
 'make',
 'place',
 'strange',
 'they',
 'tribeca',
 'volkswagen',
 'where']
\end{lstlisting}

\begin{lstlisting}[language=Python,numbers=left,xleftmargin=20pt,xrightmargin=5pt,belowskip=5pt,aboveskip=5pt]
bag_of_words
\end{lstlisting}

\begin{lstlisting}[language={},postbreak={},numbers=none,xrightmargin=7pt,breakindent=0pt,aboveskip=5pt,belowskip=5pt]
<3x12 sparse matrix of type '<class 'numpy.int64'>'
	with 13 stored elements in Compressed Sparse Row format>
\end{lstlisting}

\begin{lstlisting}[language=Python,numbers=left,xleftmargin=20pt,xrightmargin=5pt,belowskip=5pt,aboveskip=5pt]
bag_of_words.toarray()
\end{lstlisting}

\begin{lstlisting}[language={},postbreak={},numbers=none,xrightmargin=7pt,breakindent=0pt,aboveskip=5pt,belowskip=5pt]
array([[1, 0, 0, 0, 1, 0, 0, 0, 0, 0, 0, 0],
       [0, 0, 0, 1, 0, 0, 1, 1, 0, 1, 0, 0],
       [0, 1, 1, 1, 0, 1, 0, 0, 1, 0, 1, 1]], dtype=int64)
\end{lstlisting}

\begin{lstlisting}[language=Python,numbers=left,xleftmargin=20pt,xrightmargin=5pt,belowskip=5pt,aboveskip=5pt]
count.get_feature_names()
\end{lstlisting}

\begin{lstlisting}[language={},postbreak={},numbers=none,xrightmargin=7pt,breakindent=0pt,aboveskip=5pt,belowskip=5pt]
['cardi',
 'cars',
 'germany',
 'is',
 'like',
 'make',
 'place',
 'strange',
 'they',
 'tribeca',
 'volkswagen',
 'where']
\end{lstlisting}

\begin{lstlisting}[language=Python,numbers=left,xleftmargin=20pt,xrightmargin=5pt,belowskip=5pt,aboveskip=5pt]
count_2gram = CountVectorizer(ngram_range = (1, 2), stop_words="english", 
                             vocabulary=['cardi'])
\end{lstlisting}

\begin{lstlisting}[language=Python,numbers=left,xleftmargin=20pt,xrightmargin=5pt,belowskip=5pt,aboveskip=5pt]
bag = count_2gram.fit_transform(text_data)
\end{lstlisting}

\begin{lstlisting}[language=Python,numbers=left,xleftmargin=20pt,xrightmargin=5pt,belowskip=5pt,aboveskip=5pt]
bag.toarray()
\end{lstlisting}

\begin{lstlisting}[language={},postbreak={},numbers=none,xrightmargin=7pt,breakindent=0pt,aboveskip=5pt,belowskip=5pt]
array([[1],
       [0],
       [0]])
\end{lstlisting}

\subsubsection{Tfidf}\label{tfidf}

\begin{lstlisting}[language=Python,numbers=left,xleftmargin=20pt,xrightmargin=5pt,belowskip=5pt,aboveskip=5pt]
from sklearn.feature_extraction.text import TfidfVectorizer
\end{lstlisting}

\begin{lstlisting}[language=Python,numbers=left,xleftmargin=20pt,xrightmargin=5pt,belowskip=5pt,aboveskip=5pt]
tfidf = TfidfVectorizer()
feature_matrix = tfidf.fit_transform(text_data)
\end{lstlisting}

\begin{lstlisting}[language=Python,numbers=left,xleftmargin=20pt,xrightmargin=5pt,belowskip=5pt,aboveskip=5pt]
feature_matrix
\end{lstlisting}

\begin{lstlisting}[language={},postbreak={},numbers=none,xrightmargin=7pt,breakindent=0pt,aboveskip=5pt,belowskip=5pt]
<3x12 sparse matrix of type '<class 'numpy.float64'>'
	with 13 stored elements in Compressed Sparse Row format>
\end{lstlisting}

\begin{lstlisting}[language=Python,numbers=left,xleftmargin=20pt,xrightmargin=5pt,belowskip=5pt,aboveskip=5pt]
feature_matrix.toarray()
\end{lstlisting}

\begin{lstlisting}[language={},postbreak={},numbers=none,xrightmargin=7pt,breakindent=0pt,aboveskip=5pt,belowskip=5pt]
array([[0.70710678, 0.        , 0.        , 0.        , 0.70710678,
        0.        , 0.        , 0.        , 0.        , 0.        ,
        0.        , 0.        ],
       [0.        , 0.        , 0.        , 0.40204024, 0.        ,
        0.        , 0.52863461, 0.52863461, 0.        , 0.52863461,
        0.        , 0.        ],
       [0.        , 0.38988801, 0.38988801, 0.29651988, 0.        ,
        0.38988801, 0.        , 0.        , 0.38988801, 0.        ,
        0.38988801, 0.38988801]])
\end{lstlisting}

\begin{lstlisting}[language=Python,numbers=left,xleftmargin=20pt,xrightmargin=5pt,belowskip=5pt,aboveskip=5pt]
tfidf.vocabulary_
\end{lstlisting}

\begin{lstlisting}[language={},postbreak={},numbers=none,xrightmargin=7pt,breakindent=0pt,aboveskip=5pt,belowskip=5pt]
{'cardi': 0,
 'cars': 1,
 'germany': 2,
 'is': 3,
 'like': 4,
 'make': 5,
 'place': 6,
 'strange': 7,
 'they': 8,
 'tribeca': 9,
 'volkswagen': 10,
 'where': 11}
\end{lstlisting}

	\end{document}

